\documentclass[11pt]{article}

\usepackage{graphicx}
\usepackage{wrapfig}
\usepackage{url}
\usepackage{wrapfig}
\usepackage{hyperref} 
\usepackage{color}

\oddsidemargin 0mm
\evensidemargin 5mm
\topmargin -20mm
\textheight 240mm
\textwidth 160mm

\parskip 12pt 
\setlength{\parindent}{0in}

\pagestyle{myheadings} 

\title{Neural Networks for Predictive Ballistic Targeting}

\author{Vikram Chandrashekhar (vchandr6, vchandr6@jhu.edu), Ran Liu (rliu14, rliu14@jhu.edu)}
\date{12/7/2015}

\begin{document}
\maketitle

\section{Introduction and Background}
% Clearly explain your idea.
The task of ballistic targeting is such- given some input, e.g. the target's current position and velocity, one must choose a firing solution such that at some time in the future, the shot fired will intercept the target's future location. This problem is very relevant to many military applications including unmanned/manned drones, tanks, and fighter jets, among others.

There are many non-machine learning (i.e linear, circular targeting) and many machine learning methods (i.e neural nets, self-organizing maps, etc) to currently solve this task. The current, "gold standard" for targeting methods are to compute trigonometric firing solutions using the velocity and position of the target. The two trigonometric firing solutions are linear and circular targeting; as the names imply, the linear targeting method assumes that the target will move at a constant velocity in a straight line and the circular targeting method assumes the same except in a circle.



We believe that a non-linear machine learning method might be able to provide similar, if not better, results; in addition a machine learning algorithm would provide wider applicability since it will not be constrained by an assumption on the motion of the target (straight or circular motion). Therefore, we will implement a neural network with various hidden layers and nodes; in order to optimize these parameters, we will test performance of this algorithm with various neural network structures. 

\section{Machine Learning Techniques}
% Explain the methods you will be using and why they are appropriate.
Since there may not necessarily be a linear relationship between the predictors and the output, and therefore, we require a method suitable for non linearly-separable data. Furthermore, there may not be any one optimal hypothesis for classifying all data - concept drift may or may not be encountered. Intuitively, the target may change strategies in evasion. Therefore, we require an online learning method.

We propose to use an artificial neural network to solve the targeting problem, as it is capable both of online learning and of classifying nonlinear data. The task at hand would be to implement a neural network algorithm that solves this task, and compare its performance against the baseline of existing non-machine learning targeting methods, e.g. linear, circular. The performance of these algorithms would then be measured using sets of targets with predefined movement patterns, as well human-controlled targets.

The features used by the neural network will be position and velocity at each time step (including a 5-15 time step history) to predict the future motion of a target. The position and velocity will be converted to coordinates relative to the location of the turret. The relative position and velocity will then be fed into the neural network. After creating a list of possible locations of the target that cover the range of all possible locations within the bullet travel time, the output of the neural network will be the best firing solution from the candidates.

\section{Description of Work Done}
% What resources will you use and how will you get them?
We will write up our algorithm with a GUI using Swing in Java. With regards to implementation of the algorithm, we will use Apache Commons Math to represent matrices and perform matrix operations as necessary with the neural network.

We also have data for 2 test cases: linear and circular. The linear test case contains data from a target moving in a linear trajectory using randomly defined start and end points. We have generated multiple linear test cases using different start and end points. The circular test case contains data from the target moving in a circular trajectory for a finite amount of time with randomly chosen center and radius. Multiple circular motion data sets were generated as well.

Another set of test data will use combinations of the linear and circular cases of data to make combinations of movements that involve both linear and circular motion to make the target more unpredictable.

We will also be generating a human test case which will consist of data generated by someone moving the target using a mouse to avoid fire from the turret. This data will be generated once we have the GUI to visualize the turret, bullet, and target.


\section{Results}
\subsection{Must achieve}

We will implement multiple neural networks for predictive AI targeting for one turret and one target including all the functions that are required for training and testing. The networks will vary in depth and number of nodes and will be compared based on their accuracy.

\subsection{Expected to achieve}

We expect to create a GUI using Swing in Java that allows one to visualize the motion of the target and the bullets of the turret in real time.

\subsection{Would like to achieve}

We hope to implement a neural network for predictive targeting with multiple targets and turrets using the same neural network structure as for one target and one turret. However, we would like to explore new features outside of position and velocity at each time step. We would also like to modify our GUI in Swing to visualize this new situation.

\section{Comparison to Proposal}
% What will appear in the final writeup.

The final writeup will include detailed derivations for the math involved in the neural networks we chose, comparisons of the accuracies of the various neural network structures explored, and the performance of our optimum choice of neural network relative to linear and circular targeting.

We will also include screenshots at various time steps of the GUI used to display the target, turret, and bullet.

\begin{thebibliography}{1}
\bibitem{tank} Hans W. Guesgen et al. {\em An Artificial Neural Network 
for a Tank Targeting System.}  2006.

\bibitem{impj} Matthieu J.S.  {\em Imperial Japan 1800-1945} 1973:
Random House, N.Y.

\bibitem{norman} E. H. Norman {\em Japan's emergence as a modern
state} 1940: International Secretariat, Institute of Pacific
Relations.

\bibitem{fo} Bob Tadashi Wakabayashi {\em Anti-Foreignism and Western
Learning in Early-Modern Japan} 1986: Harvard University Press.

\end{thebibliography}

\end{document}
